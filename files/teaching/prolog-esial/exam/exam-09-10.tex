\documentclass[a4paper]{article}
\usepackage[utf8x]{inputenc}
\usepackage[lastexercise]{exercise}
\usepackage{color}
\usepackage[absolute]{textpos} 
\usepackage{verbatim}
\usepackage{tikz}
\usetikzlibrary{calc}
\usetikzlibrary{positioning}

\renewcommand\ExerciseName{Exercice}
\renewcommand{\AnswerHeader}{\medskip\centerline{\textbf{Solution de
                        l'\ExerciseName  \ExerciseHeaderNB}\smallskip}}
\newenvironment{CAnswer}{\color{red}\begin{Answer}}
                        {\end{Answer}}


\title{Partiel de Programmation Logique}
%\date{}

\begin{document}
\maketitle
\begin{textblock*}{4cm}(10mm,10mm)
\begin{Large}ESIAL 3A IL LE\end{Large}
\end{textblock*}
\begin{textblock*}{3cm}(160mm,10mm)
\includegraphics[width=\textwidth]{../ESIAL.pdf}
\end{textblock*}

\section*{Mise en route:}

La durée de l'examen est de \textbf{2h}. \textbf{Les exercices sont
indépendants} et peuvent être faits dans le désordre. \textbf{Tous
documents papier autorisés, l'échange de documents est interdit}.

\textbf{Certaines indications vous suggèrent de créer certains prédicats
intermédiaires.} Il n'est pas obligatoire de les utiliser et \textbf{vous
pouvez introduire vos propres prédicats auxiliaires}. Dans ce cas vous 
n'hésiterez pas à décrire leur fonctionnement et la manière de s'en
servir.

Nous considérons l'implantation Gnu-Prolog vu en cours et en TD.

{\Large Vous séparerez sur des copies différentes les 2 parties de l'examen.}

La notation tiendra compte de la présence d'explications dans vos réponses (sans
pour autant paraphraser les programmes écrits) ainsi que de la qualité de la
rédaction et de la présentation. Le barème sur 21 est donné à titre indicatif.

\begin{Exercise}[title={Résolution}]
\verbatiminput{reso.pro}
\begin{verbatim}

\Question Résolvez les assertions suivantse :
\begin{verbatim}
1-1 ?- P(2,Y).
\end{verbatim}
\begin{CAnswer}
réponse
\end{CAnswer}
\end{Exercise}

\begin{Exercise}[title={Arbres binaires}]
\verbatiminput{reso.pro}
\begin{verbatim}
Soit la structure d'arbre binaire définie comme suit:
\verb#bin(a,bin(b,nil,nil),nil)#

Soit le predicat 

taille(bin(_,X,Y)) :- .

entier de peano
add(s(X), Y, Z):- add(X, s(Y), Z).

\Question Résolvez les assertions suivantse :
\begin{verbatim}
1-1 ?- P(2,Y).
\end{verbatim}
\begin{CAnswer}
réponse
\end{CAnswer}
\end{Exercise}
\end{document}

